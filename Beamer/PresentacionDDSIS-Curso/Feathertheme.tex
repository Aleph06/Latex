\documentclass[10pt]{beamer}
\usetheme[
%%% option passed to the outer theme
%    progressstyle=fixedCircCnt,   % fixedCircCnt, movingCircCnt (moving is deault)
  ]{Feather}
  
% If you want to change the colors of the various elements in the theme, edit and uncomment the following lines
\definecolor{Accentlue}{RGB}{83,121,170}
\definecolor{BackBlue}{RGB}{3, 42, 104}
% Change the bar colors:
\setbeamercolor{Feather}{fg=Accentlue,bg=BackBlue}

% Change the color of the structural elements:
%\setbeamercolor{structure}{fg=red}

% Change the frame title text color:
%\setbeamercolor{frametitle}{fg=blue}

% Change the normal text color background:
\setbeamercolor{normal text}{fg=black,bg=gray!10}
\setbeamercolor{title}{fg=Accentlue}

%-------------------------------------------------------
% INCLUDE PACKAGES
%-------------------------------------------------------

\usepackage[utf8]{inputenc}
\usepackage[spanish,mexico]{babel}
\def\spanishoptions{mexico,mexico-com}
\usepackage[T1]{fontenc}
\usepackage{helvet}
\usepackage{graphicx} % Allows including images
\graphicspath{ {imagenes/} }
\usepackage{listings}


%-------------------------------------------------------
% DEFFINING AND REDEFINING COMMANDS
%-------------------------------------------------------

% colored hyperlinks
\newcommand{\chref}[2]{
  \href{#1}{{\usebeamercolor[bg]{Feather}#2}}
}

%-------------------------------------------------------
% INFORMATION IN THE TITLE PAGE
%-------------------------------------------------------

\title[] % [] is optional - is placed on the bottom of the sidebar on every slide
{ % is placed on the title page
      \textbf{Web Services C\# .NET}
}

\subtitle[The Feather Beamer Theme]
{
      \textbf{v. 1.0.0}
}

\author[calrodriguez]
{      Carlos Leonardo Rodríguez González \\
      {\ttfamily carlos.leonardo.rodriguez@gmail.com}
}

\institute[]
{
      Desarrollo y Diseño de Sistemas\\
      ``DDSIS''\\
  
  %there must be an empty line above this line - otherwise some unwanted space is added between the university and the country (I do not know why;( )
}

\date{\today}

%-------------------------------------------------------
% THE BODY OF THE PRESENTATION
%-------------------------------------------------------

\begin{document}

%-------------------------------------------------------
% THE TITLEPAGE
%-------------------------------------------------------

{\4% % this is the name of the PDF file for the background
\begin{frame}[plain,noframenumbering] % the plain option removes the header from the title page, noframenumbering removes the numbering of this frame only
  \titlepage % call the title page information from above
\end{frame}}


\begin{frame}{Contentenido}{}
\tableofcontents
\end{frame}

%-------------------------------------------------------
\section{Prerequisitos}
%-------------------------------------------------------
\subsection{Descargar elementos}
\begin{frame}{Prerequisitos}{Descargar elementos}
	%-------------------------------------------------------
	\begin{itemize}
		\item<1-> La \'{u}ltima versi\'{o}n disponible de Visual Studio es la 2017, existen diferentes distribuciones: Enterprise, Professional, Community, etc.
		\item<2-> Para efectos prácticos se recomienda la distribución \chref{https://www.visualstudio.com/es/downloads/}{Community}. Tras 30 días de uso solicitará una licencia la cual puede ser ``obtenida'' a traves de una cuenta de microsoft indicando que se es desarrollador independiente.
		\item<3-> Opcionales: Instalar \chref{https://www.microsoft.com/en-us/download/details.aspx?id=48264}{iis express}.
	\end{itemize}
\end{frame}
%-------------------------------------------------------
\subsection{Instalación}
\begin{frame}{Prerequisitos}{Instalación}
	%-------------------------------------------------------
	\begin{itemize}
		\item<1-> El asistente de instalción para Visual Studio es muy sencillo, sin embargo, para iniciar con el desarrollo hay que instalar ciertas herramientas para el desarrollo de Web Services.
		\item<2-> Es importante seleccionar las siguientes.
		\begin{enumerate}
			\item .NET Desktop Development (Desarrollo de escritorio)
			\item ASP.NET (Desarrollo de web).
		\end{enumerate}
		\item<3-> Opcionales: El instador de iis solo es necesario seguir el asistente.
	\end{itemize}
\end{frame}

%-------------------------------------------------------
\section{Web Service Básico}
%-------------------------------------------------------
\subsection{Crear proyecto}
\begin{frame}{Web Service Básico}{Crear proyecto}
	%-------------------------------------------------------
	\begin{itemize}
		\item<1-> Crear proyecto MiPrimerWS.
		\item<1-> Selecccionar.
	\end{itemize}
\end{frame}
%-------------------------------------------------------
\subsection{Probar WS}
\begin{frame}{Web Service Básico}{Probar WS}
	Texto para probar WS
\end{frame}
%---------
\begin{frame}{Web Service Básico}{Resultado de prueba}
	Texto
\end{frame}

%-------------------------------------------------------
\section{Web Service tipos complejos}
\subsection{Contracts}
\begin{frame}{Web Service tipos complejos}{Contracts}
	%-------------------------------------------------------
	Texto
\end{frame}
%--------------------------------
%-------------------------------------------------------
\section{Asegurar Web Service}
\subsection{Tipos}
\begin{frame}{Asegurar Web Service}{Tipos}
	%-------------------------------------------------------
	Texto
\end{frame}
\subsection{Ejemplo}
\begin{frame}{Asegurar Web Service}{Ejemplo}
	%-------------------------------------------------------
	Texto
\end{frame}
%------------
%-------------------------------------------------------
\section{Cliente WebService}
\subsection{Service Reference}
\begin{frame}{Cliente WebService}{Service Reference}
	%-------------------------------------------------------
	Texto
\end{frame}
\subsection{Ejemplo}
\begin{frame}{Cliente WebService}{Ejemplo}
	%-------------------------------------------------------
	Texto
\end{frame}
%------------
%-------------------------------------------------------
\section{Otros}
\subsection{Base de datos}
\begin{frame}{Otros}{Base de datos}
	%-------------------------------------------------------
	Texto
\end{frame}
\subsection{Otros}
\begin{frame}{Cliente WebService}{Web Deploy}
	%-------------------------------------------------------
	Texto
\end{frame}
\begin{frame}[plain,noframenumbering]
	\finalpage{DDIS $_{\text{\copyright}}$ }
\end{frame}

\end{document}